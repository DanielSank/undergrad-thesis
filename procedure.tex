\documentclass{report}

\begin{document}

\chapter{Experimental Setup}

\section{Optics}

The basic laser tweezers setup we use is similar to that described in (3). We describe here only the differences.

See figure 5 for a schematic of our optics. We use two linearly polarized lasers which we will refer to as ``A'' and ``B.'' Laser B is deflected from two mirrors before entering the main optics train. The second of these mirrors, the one closer to the beamsplitter, is in a conjugate plane to the back of the microscope objective and can be used to position the trap caused by laser B. Laser A cannot be steered. The polarizations of the lasers are rotated by ninety degrees from one another. This allows us to separate the beams after the sample cell with a polarizing beamsplitter and image each one separately onto a quadrant photodiode. Polarizers in front of each laser allow us to control the amount of power delivered to each optical trap. Note that this method of power adjustment causes no problems of mixing laser polarizations because both beams are passed through a polarizing beamsplitter before entering the optics train.

When a bead is in a trap, say trap A, the total laser power entering photodiode A is greater than with no bead. We use this to our advantage in collecting the bead hopping signal. While the bead is hopping between traps we monitor the total power entering one of the photodiodes. This signal goes high when the bead is in trap A and low when the bead is in trap B. By fitting a square wave to this signal we obtain our desired hopping signal. See figure 6 for an example of such a trace.

A camera is focused through two dichroic mirrors into the sample cell for video aquisition. The dichroic mirror directly in front of the camera is of the type that absorbs our laser light and passes all other frequencies. This is used simply to block images of the trapping lasers which are so bright that they wash out the desired image. The other mirror is used for beam control and is therefore of the type that reflects our laser light and passes other frequencies. See figure 5.

\section{Sample Cell}

Two types of cell are possible. We describe first a simple observation cell and then a more complicated flow cell which allows the sample to be injected in situ. An advantage of using the flow cell will be discussed.

See figure 4 for a picture of the simple cell. We give here instructions for its construction.

Fold a piece of parafilm over itself so that you are working with a doublethick sheet. Cut a one inch by one inch square from the sheet and then cut another smaller square from its center. Place this ``square annulus'' of parafilm on a cleaned microscope slide and place the slide on a hotplate at about 50 degrees C. The parafilm will melt and stick to the slide. Be sure that the parafilm is flat against the face of the slide. The handle of any metal tool works well to push the parafilm agains the slide. Once the parafilm has melted remove the slide from the hotplate and allow it to cool. Drop the sample of suspended beads into the center of the square hole in the parafilm making sure that the drop is not large enough that the liquid touches the parafilm. If the water does touch the parafilm it may be wicked out of the desired observation area. Clean a thin microscope coverslip, place it over the drop in the center of the parafilm square, and press it firmly against the parafilm so that it sticks. Don't try to melt the parafilm to the coverslip as the heat will cause bubbles to form in the sample. The sample is now ready to be inserted into the mount.

This type of cell is less desireable because of the requirement that the liquid not touch the parafilm. This allows the water to move in the plane transverse to the laser beams which causes some complications in consideration of the applied driving force as discussed in the next chapter.

We used the flow cell scheme in our experiments. The flow cell is constructed similarly to the simple cell but rather than constructing a square cavity for the bead suspension we form a channel, approximately two inches long and 5 millimeters wide through. The bead suspension may be injected into this channel using thin tubes and syringes. The advantage to this scheme is that the microscope slide-parafilm-coverslip sandwich can be heated and melted together before the sample is present. This allows the sample to touch the walls of the channel without being wicked away by the parafilm. This greatly simplifies the hydrodynamics of the water in the cell when the cell is oscillated. This point is discussed in the next chapter.

\end{document}