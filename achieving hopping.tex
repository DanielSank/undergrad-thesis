\Chapter{Achieving Hopping}

In this section we give a general description of how to achieve bead hopping between the two potential minima. Keep in mind this is the most reliable procedure that I have found. There may be, and indeed probably are, many ways to improve upon it.

The signature of hopping is a square wave signal in the total power delivered to either of the photodiodes. If at any time one of the photodiode signals appears to vaguely resemble a square wave try adjusting the position of that photodiode as it is possible that the diode is not at the optimal position to detect the desired signal. While trying to achieve hopping it is best to be able to simultaneously monitor both photodiode voltages as well as video of the traps and beads.

Begin with both polarizers set to maximize beam transmission and with the dichroic mirror in front of the camera. Without any beads near the trapping lasers adjust the translation stage along the z axis so that the laser foci are near the inside of the coverslip. It should be obvious when this happens because the foci become easily visible on the CCD camera. Turn the steering mirror of laser B until the two foci are approximately one quarter of a bead radius apart or slightly less. Now turn polarizer B to minimize the power in trap B. Move the stage until a bead is found at a suitable depth for trapping and trap it in laser A which is still set to full power. Turn polarizer A until the power is sufficiently low that the trapped bead's brownian motion is just large enough to be seen on camera. Now slowly turn polarizer B to increase the power in trap B. If the cosmos are correctly aligned you may get bead hopping at this point, but chances are that further tweaking will be necessary.

A typical situation at this point is the following. The power in laser B is increased until the bead suddenly jumps from trap A into trap B where it remains without making further transitions. Blocking laser B causes the bead to hop back to A and subsequently unblocking B causes the bead to hop back into trap B. This would indicate that B is too strong for hopping to occurr. In this case a slight reduction of the power in trap B may be needed. Turn polarizer B to reduce the power in B and repeat the experiment of blocking the lasers to see which trap is stronger. If the bead always returns to one particular trap when both lasers are unblocked then either reduce that trap's power or increase the power of the other. If the position of the trapped bead with both lasers unblocked is midway between the two traps (and the bead is not hopping) try moving trap B slightly farther away from trap A.

Beware that the hopping system is much more fragile than a bead in a single trap; whereas a bead trapped in a single laser will remain trapped even if the observation cell is bumped by a hand, a slight bump against the optics table may be enough to knock a bead out of the hopping potential.

This strategy worked fairly reliably several times before, for an unknown reason, we ceased to be able to acheive hopping. I believe this to be due to some change in the optics train as several components were unintentionally moved at one point after a sucessful hopping experiment. We have not been able to achieve hopping since then. In our most sucessful hopping the potential minima were approximately 0.35$\mu$m apart as could be seen on video.